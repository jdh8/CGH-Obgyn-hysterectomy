\documentclass{beamer}
\usepackage[no-math]{fontspec}
\usepackage{xeCJK}
\setCJKmainfont{Source Han Sans TW}
\hypersetup{colorlinks,linkcolor=}

\usetheme{CambridgeUS}
\title[Hysterectomy]{Total versus subtotal hysterectomy for benign gynaecological conditions}
\subtitle{
    Lethaby A, Mukhopadhyay A, Naik R \\
    \textit{Cochrane Database Syst Rev}.  2012 Apr 18
}
\author[Chen-Pang He]{何震邦 (Chen-Pang He), Intern}
\institute[CGH]{Cathay General Hospital}
\date{January 29, 2019}

\newcommand*{\solo}[1]{\centering\includegraphics[width=\textwidth, height=0.8\textheight, keepaspectratio]{#1}}

\begin{document}
\maketitle

\begin{frame}{PICO process}
    \begin{description}
        \item[Problem]      Benign gynaecological tumors
        \item[Intervention] Subtotal hysterectomy
        \item[Comparison]   Total hysterectomy
        \item[Outcome]      Urinary, bowel, and sexual functions
    \end{description}
\end{frame}

\section{Background}
\subsection{Description of the intervention}
\begin{frame}{Description of the intervention}
    \begin{itemize}
        \item A total hysterectomy involves the removal of both the uterine body and the cervix.
        \item A subtotal hysterectomy involves the removal of only the uterine body.
    \end{itemize}
\end{frame}

\begin{frame}{Rise of total hysterectomy}
    \begin{itemize}
        \item Subtotal abdominal hysterectomy remained the operation of choice until 1929.
        \item Total hysterectomy almost completely replaced subtotal hysterectomy.
            \begin{itemize}
                \item First total abdominal hysterectomy by Richardson in 1929
                \item Subsequent concerns over the potential for the development of cancer in a conserved cervix
                \item Improvements in operative and anaesthetic techniques
            \end{itemize}
    \end{itemize}
\end{frame}

\begin{frame}{Arguments for subtotal hysterectomy}
    \begin{itemize}
        \item By causing less damage to the nerves supplying the vagina,
              bladder and bowel, it is possible that subtotal hysterectomy
              might cause fewer urinary, bowel and sexual symptoms.
            \begin{itemize}
                \item These proposed benefits of subtotal hysterectomy need to
                      be reviewed and compared to outcomes with the standard
                      procedure of total hysterectomy.
            \end{itemize}
        \item The risk of cervical stump carcinoma in women with a previously normal Pap smear is no more than 0.3\%.
            \begin{itemize}
                \item Approximately the same risk as for vaginal carcinoma after hysterectomy for a benign condition
            \end{itemize}
    \end{itemize}
\end{frame}

\subsection{How the intervention might work}
\begin{frame}{How the intervention might work}
    Subtotal hysterectomy requires less dissection of surrounding tissue than
    total hysterectomy. Thus, there has been a suggestion it might be
    associated with:

    \begin{itemize}
        \item A reduced risk of damage to the bladder and ureter
        \item A reduced risk of a post-operative pelvic haematoma
        \item A reduced risk of prolapse after surgery
        \item Better sexual function
        \item Less damage to neuro-anatomical structures compared to
              total hysterectomy, thereby preserving the nerve supply to
              vagina, bladder and bowel sphincters.
    \end{itemize}
\end{frame}

\section{Objectives}
\begin{frame}{Objectives}
    To compare short term and long term outcomes of subtotal hysterectomy with
    total hysterectomy for benign gynaecological conditions.
\end{frame}

\section{Methods}
\subsection{Criteria for considering studies for this review}
\begin{frame}{Types of studies}
    Randomised controlled trials (RCTs) where subtotal hysterectomy is compared
    with total hysterectomy, by any approach (laparoscopic, abdominal or
    vaginal).
\end{frame}

\begin{frame}{Types of participants}
    \begin{block}{Inclusion criteria}
        Women undergoing hysterectomy for benign gynaecological conditions.
        Subgroup analysis will be performed according to the indication for
        hysterectomy, if there are sufficient trials.
    \end{block}

    \begin{block}{Exclusion criteria}
        Women with gynaecological cancer.
    \end{block}
\end{frame}

\begin{frame}{Types of outcome measures}
    Studies were only included if they assessed one or more of the primary
    outcomes.
\end{frame}

\begin{frame}{Primary outcomes}
    \begin{enumerate}
        \item Urinary function
            \begin{itemize}
                \item Stress incontinence
                \item Urinary urgency
                \item Voiding dysfunction
            \end{itemize}
        \item Bowel function
            \begin{itemize}
                \item Constipation
                \item Incontinence (stool)
            \end{itemize}
        \item Sexual function
            \begin{itemize}
                \item Pain symptoms or dyspareunia
                \item Satisfaction, relationship and functioning combined
            \end{itemize}
    \end{enumerate}
\end{frame}

\begin{frame}{Secondary outcomes}
    \begin{enumerate}
        \item Quality of life
        \item Operating time
        \item Recovery from surgery
        \item Short term complications (pre-discharge)
        \item Intermediate term complications (post-discharge, up to 2 y post-surgery)
        \item Long term complications (> 2 y post-surgery)
        \item Alleviation of pre-surgery symptoms
        \item Readmission to hospital
    \end{enumerate}
\end{frame}

\subsection{Data collection and analysis}
\begin{frame}{Assessment of risk of bias}
    Risk of bias was assessed independently by two review authors (AL and AM)
    during the 2011 review, using the risk of bias tool developed by Julian
    Higgins.
\end{frame}

\begin{frame}{Methodological quality}
    \solo{F1.png}
\end{frame}

\begin{frame}{Risk of bias}
    \solo{F2.png}
\end{frame}

\section{Results}
\subsection{Included studies}
\begin{frame}{Included studies}
    For the 2011 update, 9 trials met the inclusion criteria and were included
    in the review.
    
    The 9 trials randomised a total of 1553 women, but not all participants
    were included in the analysis of every outcome.
\end{frame}

\begin{frame}{Study design}
    \begin{itemize}
        \item All trials were randomised parallel group studies.
        \item Five trials were undertaken in a single centre.
        \item One trial involved 11 different centres in Denmark.
        \item One trial involved eight different centres in Sweden.
        \item One trial had four different centres in the US.
        \item One trial had two different centres in the UK.
    \end{itemize}
\end{frame}

\begin{frame}{Participants}
    \begin{itemize}
        \item Two of the studies specified that participants needed to be 30--50 y and pre-menopausal.
        \item One study required participants to be < 60 y.
        \item One study enrolled only pre-menopausal women.
        \item One study required participants to 18--55 y.
        \item One study accepted participants aged < 75 y.
        \item Two studies did not mention age criteria.
        \item The mean age of the women in the trials varied from 42 to 49 years.
        \item Two trials excluded women with known endometriosis.
    \end{itemize}
\end{frame}

\begin{frame}{Interventions}
    \begin{itemize}
        \item Six studies compared total abdominal hysterectomy with subtotal abdominal hysterectomy.
        \item For two studies the procedures were performed laparoscopically.
        \item For one study the decision whether to use an abdominal, vaginal or laparoscopic approach was left to the surgeon.
    \end{itemize}
\end{frame}

\subsection{Summary}
\begin{frame}{Summary of findings}
    \only<1>{\solo{T1.eps}}
    \only<2>{\solo{T2.eps}}
\end{frame}

\subsection{Primary outcomes}
\begin{frame}{Analysis}
    \solo{A1.eps}
\end{frame}

\begin{frame}{Analysis}
    \solo{A2.eps}
\end{frame}

\begin{frame}{Analysis}
    \solo{A3.eps}
\end{frame}

\begin{frame}{Analysis}
    \solo{A4.eps}
\end{frame}

\begin{frame}{Analysis}
    \solo{A5.eps}
\end{frame}

\begin{frame}{Analysis}
    \solo{A5.eps}
\end{frame}

\begin{frame}{Analysis}
    \solo{A6.eps}
\end{frame}

\begin{frame}{Analysis}
    \solo{A7.eps}
\end{frame}

\begin{frame}{Analysis}
    \solo{A8.eps}
\end{frame}

\begin{frame}{Analysis}
    \solo{A9.eps}
\end{frame}

\begin{frame}{Analysis}
    \solo{A10.eps}
\end{frame}

\begin{frame}{Analysis}
    \solo{A11.eps}
\end{frame}

\begin{frame}{Analysis}
    \solo{A12.eps}
\end{frame}

\begin{frame}{Analysis}
    \solo{A13.eps}
\end{frame}

\begin{frame}{Analysis}
    \solo{A15.eps}
\end{frame}

\begin{frame}{Analysis}
    \solo{A16.eps}
\end{frame}

\section{Discussion}
\begin{frame}{Primary outcomes}
    This review has not demonstrated confirmatory evidence.
\end{frame}

\begin{frame}{Secondary outcomes}
    \begin{itemize}
        \item Quality of life measures after surgery did not appear to vary according to type of hysterectomy.
        \item A significant benefit of subtotal, as compared to total,
              abdominal hysterectomy was reduced operating time and reduced
              blood loss, although no differences were reported in the
              requirement for blood transfusion.
        \item These benefits were not found for the laparoscopic approach possibly because only one trial contributed data.
        \item The average difference in operation time of 12 minutes was
              statistically different unlikely to be clinically significant
              benefits.
        \item There was no evidence of any difference in recovery from surgery.
    \end{itemize}
\end{frame}

\section{Conclusions}
\begin{frame}{Implications for practice}
    \begin{itemize}
        \item Although surgery is significantly faster and blood loss reduced,
              these may not translate to clinical benefits.
        \item Post-operative febrile morbidity is reduced with subtotal
              hysterectomy but ongoing cyclical vaginal bleeding is likely to
              be increased up to a year after surgery.
        \item A consensus opinion published by the ACOG concluded that subtotal
              hysterectomy should not be recommended by the surgeon as superior
              to total hysterectomy when indicated for benign disease.
    \end{itemize}
\end{frame}
\end{document}
